\documentclass{ctexart}
\usepackage{amssymb, mathtools, amsthm, multirow}
\newtheorem{theorem}{\indent 定理}[section]
\newtheorem{definition}{\indent 定义}[section]
\newtheorem{lemma}[theorem]{\indent 引理}
\renewcommand{\proofname}{\indent\bf 证明}

\title{近世代数笔记}
\author{王承厚}
\begin{document}
\maketitle
\section{基本概念}
\subsection{集合的定义}
一般来说,集合有四种基本运算。

1. 集合的交集

\[
    A \cap B = \{ x |  x \in A \textbf{ 且 } x \in B \}
\]

2. 集合的并集

\[
    A \cup B = \{ x |  x \in A \textbf{ 或 } x \in B \}
\]

3. 集合的差集

\[
    A - B = \{ x |  x \in A \textbf{ 且 } x \notin B \}
\]

若 A,B 是有限集,记 $|A|$ 为 A 元素的个数,则

\[
    |A \cup B| = |A| + |B| - |A \cap B|
\]

容斥原理:设 $A_i, i = 1,\ldots,n$ 为某固定集合的有限子集,则

\[
    |A_1 \cup \ldots \cup A_n| = \sum_{j=1}^{n}(-1)^{j-1} \sum_{\{i_1,\ldots,i_j\} \in \{1,\ldots,n\}}|A_{i_1} \cup|
\]

\subsection{常用记号}

\begin{itemize}
    \item $\mathbb{Z}_+$: 正整数集合
    \item $\mathbb{N} = \mathbb{Z} + \{0\}$: 自然数集合
    \item $\mathbb{Z}$: 整数集合
    \item $F[X]$: F上的(一元)多项式的集合
\end{itemize}

\subsection{群的概念}

群是一个带有一个二元运算的集合,运算满足结合律,具有单位元,且任一元素具有逆元。

\section{群}

\section{环和域}

\subsection{环的定义}

\begin{definition}
    集合 R 称为(含幺)环(ring with identity),是指 R 上存在加法和乘法两张运算,且

    (1)R 关于加法是阿贝尔群,加法单位元记为 0;

    (2)R 关于乘法是含幺半群,乘法单位元记为 1;

    (3)R 加法和乘法满足\textbf{分配律}。

    如果乘法满足交换律,则称 R 为\textbf{交换环}(commutative ring)。如果 $R - \{0\}$ 是乘法阿贝尔群,则称 R 为\textbf{域}(field)。
\end{definition}

\begin{definition}
    如果 R 为环,a 既是左可逆的,又是右可逆的,则称 a 为\textbf{可逆}(invertible),也称 a 为 R 中的\textbf{单位}(unit)。
\end{definition}


\begin{lemma}
    (1)如果 a 可逆,则 a 的左逆等于右逆且唯一,记为 $a^{-1}$.

    (2)环 R 中的单位集合构成一个群,称为 R 的\textbf{单位群}(group of units),记为 $U(R)$ 或 $R^{\times}$.
\end{lemma}

\begin{definition}
    如果交换环 R 没有零因子,称 R 为\textbf{整环}(integral domain)。
\end{definition}

\begin{lemma}
    交换环 R 为整环当且仅当\textbf{消去律}成立。
\end{lemma}

\subsection{环的同态和同构}

\begin{definition}
    设 $R_1,R_2$ 为环,映射 $f: R_1 \rightarrow R_2$ 为同态,若:

    (1)$f(1) = 1$,即将乘法单位元映到单位元。
    
    (2)对任意 $g,h \in R_1$,
    \[
    f(g+h) = f(g) + f(h),\quad f(gh) = f(g)f(h)
    \]

    如 $f$ 为单射,称为\textbf{单同态},也称\textbf{嵌入}。

\end{definition}

条件(2)不能保证条件(1)成立。

\end{document}